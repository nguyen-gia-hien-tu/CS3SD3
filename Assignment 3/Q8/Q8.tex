\documentclass{article}

\usepackage[utf8]{inputenc}
\usepackage{amsmath, amssymb}
\usepackage{enumitem}


\title{CS3SD3 - Assignment 3}
\author{Hien Tu - tun1}


\begin{document}

\maketitle

\section*{Question 8}
\begin{flalign*}
  \text{Let }
  & \text{lpr}_i \text{ be local processing of reader } i \text{ where } i = 1, 2& \\
  & \text{lpw}_i \text{ be local processing of writer } i \text{ where } i = 1, 2& \\
  & \text{tr}_i \text{ be request reading of reader } i \text{ where } i = 1, 2&\\
  & \text{tw}_i \text{ be request writing of writer } i \text{ where } i = 1, 2&\\
  & \text{r}_i \text{ be reading of reader } i \text{ where } i = 1, 2&\\
  & \text{w}_i \text{ be writing of writer } i \text{ where } i = 1, 2&
\end{flalign*}
We also introduce additional boolean variables (or atomic predicates): turn=w1
(writer 1 will write), turn=w2 (writer 2 will write) and turn=r (one or both
readers will read). \\
Now states can be defined as by atomic predicates of the form:
\begin{center}
  (str1, str2, stw1, stw2, turn)
\end{center}
\begin{flalign*}
  \text{where }
  & \text{str1} \in \{\text{lpr}_1, \text{tr}_1, \text{r}_1\}
      \text{ stands for status of reader 1}& \\
  & \text{str2} \in \{\text{lpr}_2, \text{tr}_2, \text{r}_2\}
      \text{ stands for status of reader 2}& \\
  & \text{stw1} \in \{\text{lpw}_1, \text{tw}_1, \text{w}_1\}
      \text{ stands for status of writer 1}& \\
  & \text{stw2} \in \{\text{lpw}_2, \text{tw}_2, \text{w}_2\}
      \text{ stands for status of writer 2}& \\
  & \text{turn} \in \{\text{turn=w1}, \text{turn=w2}, \text{turn=r}\}
      \text{ stands for status of turns}&
\end{flalign*}
A cycle of reader 1 is
$$(\text{lpr}_1, \_, \_, \_, \_) \rightarrow (\text{tr}_1, \_, \_, \_, \_)
  \rightarrow (\text{r}_1, \_, \_, \_, \_) \rightarrow \text{back to the
  beginning}$$
A cycle of reader 2 is
$$(\text{lpr}_2, \_, \_, \_, \_) \rightarrow (\text{tr}_2, \_, \_, \_, \_)
  \rightarrow (\text{r}_2, \_, \_, \_, \_) \rightarrow \text{back to the
  beginning}$$
A cycle of writer 1 is
$$(\_, \_, \text{lpw}_1, \_, \_) \rightarrow (\_, \_, \text{tw}_1, \_, \_)
  \rightarrow (\_, \_, \text{w}_1, \_, \_) \rightarrow \text{back to the
  beginning}$$
A cycle of writer 2 is
$$(\_, \_, \text{lpw}_2, \_, \_) \rightarrow (\_, \_, \text{tw}_2, \_, \_)
  \rightarrow (\_, \_, \text{w}_2, \_, \_) \rightarrow \text{back to the
  beginning}$$
This would result in $3 \times 3 \times 3 \times 3 \times 3 = 243$ states in the
diagram. \\
However, not all combinations of atomic predicates are allowed. For example, if
the readers are reading, then no writers can write (the other reader can still
read). That is
$$\text{str1} = \text{r}_1 \Rightarrow \text{stw1} \neq \text{w}_1 \,\land\,
\text{stw2} \neq \text{w}_2$$
Similarly,
$$\text{str2} = \text{r}_2 \Rightarrow \text{stw1} \neq \text{w}_1 \,\land\,
\text{stw2} \neq \text{w}_2$$
If one of the writer is writing, then the other writer cannot write and the two
readers cannot read. That is
$$\text{stw1} = \text{w}_1 \Rightarrow \text{str1} \neq \text{r}_1 \,\land\, 
\text{str2} \neq \text{r}_2 \,\land\, \text{stw2} \neq \text{w}_2$$
and
$$\text{stw2} = \text{w}_2 \Rightarrow \text{str1} \neq \text{r}_1 \,\land\, 
\text{str2} \neq \text{r}_2 \,\land\, \text{stw1} \neq \text{w}_1$$
In total, there would be 203 states in the diagram, which is impossible to draw. \\
Safety properties:
\begin{itemize}
  \item
    \begin{flalign*}
      \text{LTL: }
      & \text{G}(\text{r}_1 \Rightarrow \neg (\text{w}_1 \,\lor\, \text{w}_2)) & \\
      & \text{G}(\text{r}_2 \Rightarrow \neg (\text{w}_1 \,\lor\, \text{w}_2)) & \\
      & \text{G}(\text{w}_1 \Rightarrow \neg (\text{r}_1 \,\lor\, \text{r}_2 \,\lor\, \text{w}_2)) & \\
      & \text{G}(\text{w}_2 \Rightarrow \neg (\text{r}_1 \,\lor\, \text{r}_2 \,\lor\, \text{w}_1)) &
    \end{flalign*}
  \item
    \begin{flalign*}
      \text{LTL: }
      & \text{AG}(\text{r}_1 \Rightarrow \neg (\text{w}_1 \,\lor\, \text{w}_2)) & \\
      & \text{AG}(\text{r}_2 \Rightarrow \neg (\text{w}_1 \,\lor\, \text{w}_2)) & \\
      & \text{AG}(\text{w}_1 \Rightarrow \neg (\text{r}_1 \,\lor\, \text{r}_2 \,\lor\, \text{w}_2)) & \\
      & \text{AG}(\text{w}_2 \Rightarrow \neg (\text{r}_1 \,\lor\, \text{r}_2 \,\lor\, \text{w}_1)) &
    \end{flalign*}
\end{itemize}
Liveness properties:
\begin{itemize}
  \item
    \begin{flalign*}
      \text{LTL: }
      & \text{G}(\text{tr}_1 \Rightarrow \text{F } \text{r}_1) & \\
      & \text{G}(\text{tr}_2 \Rightarrow \text{F } \text{r}_2) & \\
      & \text{G}(\text{tw}_1 \Rightarrow \text{F } \text{w}_1) & \\
      & \text{G}(\text{tw}_2 \Rightarrow \text{F } \text{w}_2) &
    \end{flalign*}
  \item
    \begin{flalign*}
      \text{LTL: }
      & \text{AG}(\text{tr}_1 \Rightarrow \text{AF } \text{r}_1) & \\
      & \text{AG}(\text{tr}_2 \Rightarrow \text{AF } \text{r}_2) & \\
      & \text{AG}(\text{tw}_1 \Rightarrow \text{AF } \text{w}_1) & \\
      & \text{AG}(\text{tw}_2 \Rightarrow \text{AF } \text{w}_2) &
    \end{flalign*}
\end{itemize}

\end{document}