\documentclass{article}

\usepackage[utf8]{inputenc}
\usepackage{enumitem}
\usepackage{amsmath}
\usepackage{amsthm}
\usepackage{amssymb}
\usepackage{graphicx}
\usepackage{tikz}

\title{Assignment 1}
\author{Hien Tu - tun1}
\date{\today}

\begin{document}

\maketitle

\subsection*{Question 12}
\begin{enumerate}[label=\alph*.]
    \item Show $P_2 \approx P_3$, i.e., $P_2$ and $P_3$ are \emph{bisimilar}. \\

        Clearly, $p_0 \approx s_0$ as only $a$ and $b$ come out of both $p_0$ and
        $s_0$. After trace $a$, in $P_2$, $p_0$ goes to state $p_1$; in $P_3$, 
        $s_0$ goes to state $s_1$. $p_1 \approx s_1$ since only $a$ comes out of
        both $p_1$ and $s_1$. $p_2 \approx s_2$ since $a, b$ and $c$ can be
        executed from both states. $p_3 \approx s_3$ since only $a$ and $c$ can
        come out of both $p_3$ and $s_3$. $p_4 \approx s_4$ since only $a$ and
        $b$ come out of both $p_4$ and $s_4$. $p_5 \approx s_5$ since only $c$
        can be executed from both states and both lead back to the starting
        states. $p_5 \approx s_6$ since only $c$ can be executed from both states
        and both lead back to the starting states as well. Hence $P_2 \approx P_3$. \\
    
    \item Show that $P_1 \not \approx P_2$, i.e., $P_1$ and $P_2$ are not
          \emph{bisimilar}. \\

        Clearly, $q_0 \approx p_0$ since only transition $a$ can be executed from
        both $q_0$ and $p_0$. Then $q_1 \approx p_1$ since only $a$ comes out of
        both $p_1$ and $q_1$. After this trace $a$, in $P_1$, $q_1$ goes to either
        $q_2$ or $q_3$ while $p_1$ goes to $p_2$ in $P_2$. However, $q_2 \not
        \approx p_2$ since at state $p_2$, $a$, $b$, and $c$ can be executed but
        from state $q_2$, only $a$ and $c$ can be executed. $q_3 \not \approx p_2$
        either since only $a$ and $b$ can be executed from $q_3$. Therefore,
        $P_1 \not \approx P_2$. \\

    \item Show that $P_1 \not \approx P_3$, i.e., $P_1$ and $P_3$ are not
          \emph{bisimilar}. \\

        Clearly, $q_0 \approx s_0$ since only transition $a$ can be executed from
        both states. After trace $a$, in $P_1$, $q_0$ goes to state $q_1$ while
        in $P_3$, $s_0$ goes to state $s_1$. $q_1 \approx s_1$ since only
        transition $a$ can be executed in both cases. After this trace $a$, in
        $P_1$, $q_1$ goes to either $q_2$ or $q_3$ while in $P_3$, $s_1$ goes to
        state $s_2$. However, $s_2 \not \approx q_2$ and $s_2 \not \approx q_3$.
        At $s_2$, $a$, $b$ and $c$ can be executed while at $q_2$, only $a$ and
        $c$ can be executed; while at $q_3$, only $a$ and $b$ can be executed.
        Hence, $P_1 \not \approx P_3$.

    \item Traces($P_1$) = Traces($P_2$) = Traces($P_3$)
        = Pref($(aa (c^*a \, \cup \, b^*a) c)^*$)

\end{enumerate}



\end{document}